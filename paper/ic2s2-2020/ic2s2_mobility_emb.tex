\documentclass[a4paper,12pt]{article}
\usepackage[utf8]{inputenc}
\usepackage[english]{babel}
\usepackage{authblk}
\usepackage{graphicx}
\usepackage{mathptmx}
\usepackage[singlespacing]{setspace}
\usepackage[headheight=1in,margin=1in]{geometry}
\usepackage{fancyhdr}
\usepackage[labelfont=bf, font=small]{caption}


\usepackage[style=nature,
					backend=bibtex,
					sortcites=true,
					autocite=superscript
]{biblatex}
\addbibresource{mobility.bib}

\renewcommand{\headrulewidth}{0pt}
\pagestyle{fancy}
\chead{%
  6$^{th}$ International Conference on Computational Social Science IC$^{2}$S$^{2}$\\
  July 17-20, 2020, Massachusetts Institute of Technology, United States%
}


\graphicspath{{images/}}

\title{Embedding co-affiliation trajectories reveals structure of global scientific mobility}

\author[]{} % Please leave Author-field blank for blind review and remove information that may identify the author(s)
 
\date{}

\begin{document}

\maketitle
\thispagestyle{fancy}

\vspace{-6em}
\begin{center}
\textbf{\textit{Keywords: Science of Science; Mobility; Graph embedding; Geography; Gravity Model}}
\newline
\end{center}

\section*{Extended Abstract}

Scientific mobility is closely linked to impact, innovation, and knoweldge diffusion~\autocite{sugimoto_scientists_2017, wagner_new_2008}.
Central to mobility is the notion of proximity~\autocite{torre_proximity_2005} between places.
However, geography alone is insufficient to understand scientific mobility, whereas other kinds of proximity are difficult to meaningfully measure~\autocite{boschma_proximity_2005}.
The result is a fragmented and partial understanding of scientific mobility. 

\begin{figure}[h!]
	\centering
	\includegraphics[width=\textwidth]{../Figs/ic2s2-2020/ic2s2_2020_mainfig.pdf}
	\caption{ 
	\textbf{Embedding offer robust model of co-affiliation, reveals multi-scale structure.}
	\textbf{a.} Examples of how ``sentences'' for the skip-gram model are created from author's affiliations listed on papers.
	\textbf{b.} Organization sentences are used as input for word embedding model. Each organization is represented as a vector.
	\textbf{c.} Organization embedding better captures closeness (flux) than does geographic distance. 
 $P_{i}$ and $P_{j}$ are the number of researchers in organization $i$ and $j$, and $F_{ij}$ is the symmetric flow between them.
	Black points are the mean flux for each binned by distance or similarity. 
	\textbf{d.} Cosine similarity better predictor of flux than geographic distance, using gravity law.
	Box-plots show distribution of actual flux binned across predicted, colored by the degree to which it overlaps $x = y$. 
	Cosine similarity is better predictor when considering only pairs of organizations within the same country (\textbf{e}) and only pairs in different countries (\textbf{f}). 
	\textbf{g.} UMAP visualization of embedding demonstrates geographic structure.
	Visualization of vector subsets reveal additional structure for the \textbf{h.} U.S. and \textbf{i.} Massachusetts. 
	 }
	\label{fig:image}
\end{figure}

Leveraging recent advances in neural networks and representation learning, we construct a dense, meaningful, and coherent vector-space embedding of global scientific mobility, using individual mobility trajectories documented in bibliographic data. 
This representation implicitely captures organizational and cognitive proximity between organizations~\autocite{boschma_proximity_2005}, allowing for the holistic study of scientific mobility. 

We source the Web of Science for bibliographic data on more than 22 million papers from mobile authors published between 2008 and mid 2019, comprising 3.7 million disambiguated authors and 8,445 organizations. 
A ``sentence'' is constructed for each author by concatenating organization IDs for each affiliation listed on their papers, ordered by year of publication (Fig~\ref{fig:image}.a).
These sentences are used as input into the standard word embedding~\autocite{mikolov_distributed_2013}, which learns a dense vector-space embedding of organizations (Fig~\ref{fig:image}.b) 

Similarity between vectors in the embedding space can be considered as a reasonable measure of effective distance. 
The cosine similarity between vectors can explain more than twice the variance in mobility between pairs of organizations than geographic distance (Fig~\ref{fig:image}.c). 
Moreover, applying the gravity law of mobility~\autocite{simini_universal_2012} with cosine similarity produces better predictions of mobility than does using geographic distance (Fig~\ref{fig:image}.d.
These predictions are also more robust than geographic distance when considering only mobility between organizations in the same-country (Fig~\ref{fig:image}.e) and in different countries (Fig~\ref{fig:image}.f). 

Visualizing the embedding using UMAP~\autocite{mcinnes_umap_2018} reveals organizational clusters that reflect a mix of geographic, historic, cultural, and linguistic relationships between countries (Fig~\ref{fig:image}.g). 
For example, South American countries are near Spain and Portugal and organizations in French-speaking Quebec and North Africa are nearer France. 
By visualizing subsets of vectors, we reveal more granular structure.
For example, the structure of U.S. universities demonstrates state-level geographic clustering (Fig~\ref{fig:image}.h) roughly organized by state and region, whereas states such as Massachusetts (Fig~\ref{fig:image}.i), California, and New York (not shown) instead show structure based on urban centers (Boston vs. Worcester), organization type (hospitals tend to be adjacent) and different university systems (UMass system cluster separate from Harvard, MIT, etc.).

By embedding scientific mobility, we offer a novel quantitative framework to study scientific mobility, and potentially mobility in general. 
This approach transforms complex and high-dimensional data into a dense, coherent, and meaningful vector space that is convenient for visualization, computation, and can be incorporated into a variety of analyses.



%
% Bibliography
%
%\bibliographystyle{acm}
\begingroup
\setstretch{0.8}
\setlength\bibitemsep{1pt}
{
\small
\renewcommand{\markboth}[2]{}% Remove header adjustment
\printbibliography
}
\endgroup


\end{document}
