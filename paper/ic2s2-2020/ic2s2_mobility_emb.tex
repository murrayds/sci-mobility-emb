\documentclass[a4paper,12pt]{article}
\usepackage[utf8]{inputenc}
\usepackage[english]{babel}
\usepackage{authblk}
\usepackage{graphicx}
\usepackage{mathptmx}
\usepackage[singlespacing]{setspace}
\usepackage[headheight=1in,margin=1in]{geometry}
\usepackage{fancyhdr}
\usepackage[labelfont=bf, font=small]{caption}


\usepackage[style=nature,
					backend=bibtex,
					sortcites=true,
					autocite=superscript
]{biblatex}
\addbibresource{mobility.bib}

\renewcommand{\headrulewidth}{0pt}
\pagestyle{fancy}
\chead{%
  6$^{th}$ International Conference on Computational Social Science IC$^{2}$S$^{2}$\\
  July 17-20, 2020, Massachusetts Institute of Technology, United States%
}


\graphicspath{{images/}}

\title{Embedding co-affiliation trajectories reveals structure of global scientific mobility}

\author[]{} % Please leave Author-field blank for blind review and remove information that may identify the author(s)
 
\date{}

\begin{document}

\maketitle
\thispagestyle{fancy}

\vspace{-6em}
\begin{center}
\textbf{\textit{Keywords: Science Studies; Mobility; Deep Learning; word2vec; Gravity Law }}
\newline
\end{center}

\section*{Extended Abstract}
To compete for global talent, policy-makers and administrators at all levels of governance require an understanding of \textit{scientific mobility} — the flows or researchers between organizations. 
Many interacting factors shape these flows, including geography, organizational prestigee, and policy~\autocite{deville_career_2014, clauset_systematic_2015}.
However, the complexity of scientific mobility have resulted in a fragmented understanding, limited to specific regions and mobility types.
We leverage recent advances in neural networks and representation learning in order to construct a dense, meaningful, and coherent vector-space representation of global scientific mobility, defined as researcher's co-affiliations during a time window. 
This representation facilitates the holistic study of scientific mobility, one that captures the complex relationships between organizations the world over. 

We source bibliographic data on more than 22 million papers published between 2008 and mid 2019, comprising 3.7 million disambiguated authors and 8,445 organizations from the Web of Science database hosted by Leiden University. 
A "sentence" is constructed for each author by concatenating organization ids for each affiliation listed on their papers, ordered by year of publication, as shown in Fig \ref{fig:image}.a for two example authors. 
These sentences are used as input into the standard (\textit{word2vec}) skip-gram model~\autocite{mikolov_distributed_2013}. 

Similarity between vectors in the embedding space reasonably approximate actual mobility flows, and are an effective distance. 
The cosine similarity between vectors is more strongly correlated with \textit{flux} between organizations than geographic distance (Fig~\ref{fig:image}.b). 
Flux, the flows of researchers between two organizations given their sizes, is derived from the gravity law of mobility ~\autocite{simini_universal_2012}.
Moreover, defining the gravity law as a function of cosine similarity produces better predictions of flux than using geographic distance (Fig~\ref{fig:image}.c). 

We project the embedding into a 2-dimensional space using \textit{umap} ~\autocite{mcinnes_umap_2018} and identify that organizations are largely clustered geographically (Fig ~\ref{fig:image}.d). 
By re-projecting subsets of vectors, we reveal more granular structure.
For example, a re-projection of U.S. universities again demonstrates state-level geographic structure (Fig~\ref{fig:image}.e), however re-projecting Massachusetts (Fig~\ref{fig:image}.d) reveals additional structure based on urban centers (Boston vs. Worcester), but also organization (hospitals tend to be near one another) and political organization (UMass system cluster separate from Harvard, MIT, etc.).
This embedding also reflects a mix of geographic, historical, religious, and linguistic relationships between countries, demonstrated by the complex clustering of Central and South-East Asian countries (Fig~\ref{fig:image}.g) and between Spain, Portugal, and Central and South American countries (Fig~\ref{fig:image}.h). 

By embedding scientific mobility, we offer a new way to study scientific mobility, and mobility more generally. 
This approach transforms complex and high-dimensional data into a dense, coherent, and meaningful vector space that is convenient for visualization, and computation, and can be incorporated into a variety of analyses. 


\begin{figure}[ht!]
	\centering
	\includegraphics[width=0.83\textwidth]{../Figs/ic2s2-2020/ic2s2_2020_mainfig.pdf}
	\caption{ 
	\textbf{Embedding offer robust model of co-affiliation, reveals multi-scale structure.}
	\textbf{a:} Examples of how "sentences" for the skip-gram model are created from author's affiliations listed on papers.
	\textbf{b:} Relationship between \textit{flux} and embedding similarity stronger than for geographic distance. 
	Red line is line of best fit.
	Black points are the mean flux for each binned by distance or similarity. 
	\textbf{c:} Predicting flux using gravity law.
	Box-plots show distribution of actual flux binned by expected, and color represents the the degree to which the distribution overlaps the line $x = y$. 
	\textbf{d:} Umap projection of embedding space demonstrating geographic clustering or organizations.
	Umap re-projections of subsets of organization vectors vectors for organizations in the United States (\textbf{e}), Massachusetts (\textbf{f.}), Central- and South-East Asia (\textbf{g}) and Central and South-American and Iberian countries (\textbf{h}). 
	 }
	\label{fig:image}
\end{figure}

%
% Bibliography
%
%\bibliographystyle{acm}
\begingroup
\setstretch{0.8}
\small
\setlength\bibitemsep{1pt}
\printbibliography
\endgroup


\end{document}
