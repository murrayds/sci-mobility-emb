\documentclass[12pt]{article} %{{{

% Figures
\usepackage[margin=1in]{geometry}
\usepackage{graphicx}
\def\figdir{../Figs}

% Math
\usepackage{amsmath}
\usepackage{amssymb}
\DeclareMathOperator*{\argmin}{\arg\!\min}
\DeclareMathOperator*{\argmax}{\arg\!\max}

% abbreviations
\def\etal{\emph{et~al}.\ }
\def\eg{e.g.,~}
\def\ie{i.e.,~}
\def\cf{cf.\ }
\def\viz{viz.\ }
\def\vs{vs.\ }

% Refs
\usepackage[style=nature,
					backend=bibtex,
					sortcites=true,
					autocite=superscript
]{biblatex}
\addbibresource{mobility.bib}

\usepackage{url}

\newcommand{\secref}[1]{Section~\ref{sec:#1}}
\newcommand{\figref}[1]{Fig.~\ref{fig:#1}}
\newcommand{\tabref}[1]{Table~\ref{tab:#1}}
%\newcommand{\eqnref}[1]{\eqref{eq:#1}}
%\newcommand{\thmref}[1]{Theorem~\ref{#1}}
%\newcommand{\prgref}[1]{Program~\ref{#1}}
%\newcommand{\algref}[1]{Algorithm~\ref{#1}}
%\newcommand{\clmref}[1]{Claim~\ref{#1}}
%\newcommand{\lemref}[1]{Lemma~\ref{#1}}
%\newcommand{\ptyref}[1]{Property~\ref{#1}}

% for quick author comments 
\usepackage[usenames,dvipsnames,svgnames,table]{xcolor}
\definecolor{light-gray}{gray}{0.8}
\def\del#1{ {\color{light-gray}{#1}} }
\def\yy#1{ {\color{red}\textbf{yy: #1}} }

%}}}

\begin{document} %{{{

\title{Word2vec learns effective distance of scientific co-affiliation} %{{{
\date{\today}
\maketitle %}}}

\section{Introduction}\label{sec:introduction} %{{{


%}}}

%
% DATA AND METHODS
%
\section{Data and Methods}
\label{sec:datamethods} %{{{ 

Data, analysis, and associated code have been made publicly available and can be found at \textit{https://github.com/murrayds/...}


\subsection*{Data}
%% Main dataset
% Do we need to add the Leiden field classificaiton infomration? It doesn't seem like its relevant for the current analysis.
% Do we need to talk about how we define mobility in relation to past papers?
We source co-affiliation trajectories of authors from the Web of Science database hosted by the Center for Science and Technology Studies at Leiden University. 
Trajectories are constructed from author affiliations listed on the byline of publications for a given author.
Given the limitations of author-name disambiguation, we limit to papers published after 2008, when the Web of Science began providing full names and institutional affiliations~\cite{caron2014disambiguation} and when performance is strongest (ref to SI?). 
This yields 37,542,144 author-affiliation representing 12,963,792 authors. 
Each author-affiliation combination is associated with the publication year and a unique id that maps the affiliation to a disambiguate organization. 
Trajectories are represented as the list of author-affiliation combinations, ordered by year of publication, and randomly ordered for combinations within the same year. 
Authors are classified as mobile when they have at least two distinct organization ids in their trajectory, meaning that they have published using two or more distinct affiliations between 2008 and 2019.
Mobile authors constitute 3,709,869 or 28.6 percent of all authors, and 22,436,637 author-affiliation combinations. 

% %Organization information
Affiliations mapped to one of 8,661 organizations, disambiguated following a mostly manual process originally designed for the Leiden Rankings of World Universities~\cite{waltman2012leidenrankings}. 
Organizational records were associated with a full name, a type indicating the sector (e.g., University, Government, Industry), and an identifier for the country and city of the organization. 
Sixteen different sector types were included in the analysis, which we aggregated to a four high-level codes: \textit{University}, \textit{Hospital}, \textit{Government}, and \textit{Other}. 
Each record was also associated with a latitude and longitude, however for many organizations these were missing or incorrect. We manually updated the coordinates of 2,267 organizations by searching the institution name and city on Google Maps;
in cases where a precise location of the organization could not be identified, we used the coordinates returned when searching the name of the city.
State/province level  information were added using the reverse geocoding service LocationIQ using each organization's latitude and longitude as input. 
Organizations were also assigned continental information (e.g., North America, Asia, Europe) manually, by country. 
For each organization, we also calculated a size as the average number of unique authors (mobile and non-mobile) who published with that organization across each year of our dataset;
in the case that authors publish with multiple affiliations in a single year, they are counted towards each. 


\subsection*{Embedding}
%% Mobility sentences
We embed co-affiliation trajectories by treating them analogously to sentences, and organizations analogously to words. 
The trajectories of mobile authors were converted to ''mobility sentences`` by concatenating together their ordered list of unique organization identifiers, as demonstrated in Fig.~\ref{fig:methods:mobility_sentence}.a.
In more complex cases, such as listing multiple affiliations on the same paper or publishing with different affiliations on multiple affiliations in the same year, the order is randomized within that year, as shown in  Fig.~\ref{fig:methods:mobility_sentence}.b. 
The mean length of mobility sentences is 6.0, whereas the largest is 251.
duplicate organization ids appearing in the same sentence are not removed. 


%% word2vec
These mobility sentences are used as input to the standard skip-gram negative sampling (SGNS) model for word embedding, commonly known as \textit{word2vec}~\cite{mikolov2013word2vec} . 
In natural-language processing, word embeddings are used to construct dense vector-space representations of words and phrases, in which distance relationships between words correspond to a notion of semantic distance. 
By embedding co-affiliation trajectories, we aim to learn a dense vector for every organization where distance between vectors relates to the tendency for two organizations to co-occur in a token window around the target term in the same career trajectory. 
Vectors are learned by training a neural network to optimize the prediction of so-called context term given a target, in our case using one organization to predict the others appearing in the same co-affiliation trajectory. 

%% Word2vec parameters and training procedure
The SGNS model has several tunable hyper-parameters, including the size of the learned vectors, the size of the token window  considered when training, the minimum frequency threshold for including an organization, and the number of iterations.
We iterate through several values of vector size and window size, repeating analysis for each combination. 
To mitigate the effect of less common organizations, only organizations appearing at least 50 times in mobility sentences are included in the embedding;
this yields 6,580, or 76 percent of all organizations.
We iterate the SGNS training for five epochs over all mobility sentences. 
All word embeddings are trained using the \textit{gensim} python implementation of the SGNS \textit{word2vec} model. 

%
% Distances subsection
%
\subsection{Distance}
%% Distance calculations
We calculate $f_{ij}$, the pairwise number of co-affiliations between organizations, calculated as the number of co-occurrences of organizations within author co-affiliation trajectories. 
For example, $f_{ij} = 10$ indicates that ten authors were co-affiliated with both organization $i$ and $j$ between 2008 and 2019, as evidenced from their publications. 
In contrast to traditional mobility studies, all co-affiliations are symmetric, meaning that $f_{ij} = f_{ji}$. 

We also calculate two forms of proximity between organizations. 
The geographic distance, $g_{ij}$, is the pairwise geographic distance between organizations.
Geographic distance is calculated as the great circle distance, in kilometers, between pairs of organizations. 
When less than one kilometer, a distance of one is imputed. 
The embedding proximity, $d_{ij}$, is calculated as the pairwise cosine similarity between vectors from the learned organization embedding, where $v_{i}$ and $v_{j}$ represent the vectors and $d_{ij} = \frac{v_{i} \dot v_{j}}{\lVert v_{i} \rVert \lVert v_{j} \rVert}$.












%
% Figure - Example of mobility sentences
%
\begin{figure}[ht!]
	\centering
	\label{fig:methods:mobility_sentence}
	\includegraphics[width=0.7\textwidth]{\figdir/mobility_example_combined.pdf}
	\caption{
		\textbf{Co-affiliation trajectories represented as sentences}
		\textbf{a.} 
		An authors published five papers across five time periods, with only one affiliation listed in the byline of each paper. 
		A unique identifier is assigned to each organization, and these are concatenated into a mobility sentence. 
		\textbf{b.}
		In a more complex case, an author lists several affiliations on the byline of papers in $t_{2}$, $t_{3}$, and $t_{4}$. Here, organization ids are still ordered by year, however order within a single year is random.  	
	}
\end{figure}



% Leiden bibliographic data
	% Basic details
	% How was mobility determined?
	% How were university sizes extracted?
	% Organization information? (count, etc.)
% Manually-identified geographic coordiantes
% Leiden rankings (if using prestige)
% 

%% Introduce embedding methodology and approach
% Introduce embedding and why we use it
% How were sentences created (use figure)

%% Gravity Law
% Why use gravity law?
% How were geographic distances calculated?
% How are distnaces calculated within the embedding space






%% Introduce gravity measurements

\section{Results}\label{sec:results} %{{{ 

Let $f_{ij}$ represents the total flux (the number of people who moved) between $i$ and $j$. 
Because we are ignoring the directions, $f_{ij} = f_{ji}$. 
The total flux from or to $i$ can be written as $\mu_{i} = \sum_k f_{ik}$. 
Then the conditional probability $p(j|i)$ is simply
\begin{equation}\label{fig:flux_conditional_prob}
p(j|i) = \frac{f_{ij}}{\mu_i}. 
\end{equation}

Let us assume that we have identified the best embedding vector $v_i$ for every institution. 
The skip-gram model maximizes the average log probability
\begin{equation}
\frac{1}{T}\sum_{t=1}^{T} \sum_{-c \le j \le c, j \neq 0} \log p(w_{t+j}|w_t),  
\end{equation}
where 
\begin{equation}
p(w_O|w_I) = \frac{\exp(v'_{w_{O} }\cdot v_{w_{I}})}{\sum_{w=1}^{W} \exp(v'_w \cdot v_{w_{)}}}, 
\end{equation}
where $v_w$ and $v'_w$ are `input' and `output' vector representations of $w$. 

The best embedding should be able to closely approximate the true conditional probability Eq.~\ref{fig:flux_conditional_prob}.  \yy{why? can we prove this?} 

\begin{equation}
p(j|i) = \frac{f_{ij}}{\mu_i} \simeq \frac{\exp(v'_j \cdot v_i)}{\sum_{w=1}^{W} \exp(v'_w \cdot v_i)}
\end{equation}

%\begin{minipage}{\figwidth} 
%\includegraphics[width=\linewidth]{\figdir/img} 
%\end{minipage}

%}}}

%}}}

\printbibliography{}
    
\end{document} %}}}
