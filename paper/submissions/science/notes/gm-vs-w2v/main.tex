\documentclass[12pt]{article} %{{{
\usepackage[margin=1in]{geometry}

% Figures
\usepackage{graphicx}
\graphicspath{{../../figs/}}

% Math
\usepackage{amsmath}
\usepackage{amssymb}
\DeclareMathOperator*{\argmin}{\arg\!\min}
\DeclareMathOperator*{\argmax}{\arg\!\max}
\DeclareMathOperator{\Tr}{Tr}
\DeclareMathOperator{\diag}{diag}
\def\given{\mid}


% abbreviations
\def\eg{e.g.,~}
\def\ie{i.e.,~}
\def\cf{cf.\ }
\def\viz{viz.\ }
\def\vs{vs.\ }

% math notation
\usepackage{bm}
\def\tnull{{\text{null}}}
\def\vec#1{{\bm #1}}
\def\mat#1{\mathbf{#1}}

% Refs
\usepackage{biblatex}
\addbibresource{main.bib}

\usepackage{url}

\newcommand{\secref}[1]{Section~\ref{sec:#1}}
\newcommand{\figref}[1]{Fig.~\ref{fig:#1}}
\newcommand{\tabref}[1]{Table~\ref{tab:#1}}
%\newcommand{\eqnref}[1]{\eqref{eq:#1}}
%\newcommand{\thmref}[1]{Theorem~\ref{#1}}
%\newcommand{\prgref}[1]{Program~\ref{#1}}
%\newcommand{\algref}[1]{Algorithm~\ref{#1}}
%\newcommand{\clmref}[1]{Claim~\ref{#1}}
%\newcommand{\lemref}[1]{Lemma~\ref{#1}}
%\newcommand{\ptyref}[1]{Property~\ref{#1}}

% comment out when submitting
% for quick author comments 
\usepackage[usenames,dvipsnames,svgnames,table]{xcolor}
\usepackage{soul}
\usepackage{ulem}
\definecolor{light-gray}{gray}{0.8}
\definecolor{light-blue}{RGB}{33,150,243}
\definecolor{light-green}{RGB}{76, 175, 80}
%\def\del#1{ {\color{light-gray}{#1}} }
\def\yy#1{\footnote{\color{red}\textbf{yy: #1}} }
\def\sada#1{\footnote{\color{light-blue}\textbf{sk: #1}} }
\def\jisung#1{\footnote{\color{light-green}\textbf{js: #1}} }

% Track Changes
\definecolor{myblue}{RGB}{33,150,243}
\definecolor{mygreen}{RGB}{76, 175, 80}
\definecolor{purple}{RGB}{170, 0, 255}


\newcommand{\add}[1]{{\leavevmode\color{blue}#1}}
\newcommand{\del}[1]{{\leavevmode\color{red}\sout{#1}}}
\newcommand{\change}[3][]{{\del{#2}\add{#3}$^{\textcolor{blue}{\text{#1}}}$}}
\newcommand{\comment}[1]{
    \begin{enumerate}[label*=(\arabic*)]
    #1
    \end{enumerate}
}

%}}}

\begin{document} %{{{

\title{Note on the word2vec and the gravity model} %{{{
\date{\today}
\author{Sadamori Kojaku}
\maketitle %}}}

\section{The word2vec and the euclidean distance}

With the word2vec, one writes the probability of target node $j$ given center node $i$ as 
\begin{align}
  \label{eq:model_node2vec}
    P_{\text{W2V}}\left(j \given i\right) := \frac{1}{Z_i}\exp\left(\vec{u} ^\top _{i} \vec{v}_{j}\right), 
\end{align}
where vectors $\vec{u}_i$ and $\vec{v}_i$ are the column vectors of length $K$ that represent the coordinates of $i$ and $j$ in the embedding space spanned by in-vectors and out-vectors, respectively.
Variable $Z_i$ is the normalization constant, i.e., $Z_i:= \sum_{\ell=1}^N \exp\left(\vec{u} ^\top _{i} \vec{v}_{\ell}\right)$. 
Denote by $d(\vec{u},\vec{u}')$ the distance between the in-vectors, i.e., 
\begin{align}
    \label{eq:distance}
    d(\vec{u},\vec{u}'):= \sqrt{\sum_{k=1}^K \left(u_{k} - u'_{k}\right)^2}.
\end{align}
We assume that the in-vectors are equal to the out-vectors, i.e., $\vec{v}_j = \vec{u}_j$.
Exploiting $d^2 (\vec{u},\vec{u}')=||\vec{u}||^2 + ||\vec{u}'||^2 - 2\vec{u} ^\top \vec{u}'$, the dot similarity for $\vec{u}_i$ and $\vec{v}_j$ is given by 
\begin{align}
    \label{eq:dotsim}
    \vec{u}_i ^\top \vec{v}_j = \frac{1}{2}\left[ ||\vec{u}_i||^2 + ||\vec{u}_j||^2 - d\left(\vec{u}_i, \vec{u}_j\right)^2 \right],
\end{align}
which leads 
\begin{align}
    \exp\left( \vec{u}_i ^\top \vec{v}_j \right) &= \frac{\exp\left( ||\vec{u}_i||^2/2 + ||\vec{u}_j||^2/2 \right)}{\exp\left[ d\left(\vec{u}_i, \vec{u}_j\right)^2 / 2 \right]} \nonumber \\
                                                 &= \frac{s_i s_j}{\exp\left[d\left(\vec{u}_i, \vec{u}_j\right)^2 / 2\right]}, \label{eq:exp_distance_form}
\end{align}
where $s_i:=\exp(||\vec{u} _i||^2/2)$. 
Substituting Eq.~\eqref{eq:exp_distance_form} into Eq.~\eqref{eq:model_node2vec} yields
\begin{align}
    \label{eq:model_node2vec_distance}
    P_{\text{W2V}}\left(j \given i\right) &:= \frac{1}{Z_i}\cdot \frac{s_is_j}{\exp\left[d\left(\vec{u}_i, \vec{u}_j\right)^2/2\right]}. 
\end{align}
The walkers in point $i$ tend to move to the points with a large $s_j$ more than those with a smaller $s_j$.
Thus, variable $s_j$ represents the {\it pulling force} for node $j$\sada{tentative naming}. 
It turns out that $s_j$ is a mass variable for the gravity model (see Section 2). 

\section{The word2vec is a special case of the gravity model}

The gravity model generates the flow from $i$ to $j$ by 
\begin{align}
    \label{eq:gravity_model}
    T_{ij}:= G M_i M_j f\left(d(\vec{u}_i,\vec{u}_j)\right),
\end{align}
where $M_i$ ($M_i>0$) is the mass for the gravity model, $f$ is a function for the distance, and 
$G$ is the constant analogous to the gravitational constant in physics.  

How should $M_i$ be chosen? The choice of $M_i$ is entirely unconstrained.
In migration study, one sets $M_i$ to exogenous variables such as population at the point in priori and then estimates other parameters, e.g., $G$.
We show that the word2vec automatically estimates the mass $M_i$ from data without exogenous variables. 
The estimated $M_i$ may not be equal to the number of people at a point.

We regard $P_{\text{W2V}}(j \vert i)$ as the transition probability from $i$ to $j$ in a Markov process.
Denote by $\pi_i$ the number of scientists in institution who may move to different institutions, i.e., mobile scientists.
We note that $\pi$ may not be equal to the number of scientists in the institution because some scientists always stay in the institutions.
The number of scientists that move from $i$ to $j$ (i.e., flow) in one step is given by 
\begin{align}
    \label{eq:flow}
    T_{ij}:= \pi_i P_{\text{W2V}}(j \vert i).
\end{align}
We assume that the Markov process satisfies detailed balanced condition:  
\begin{align}
    \label{eq:detailed_balanced}
    \pi_i P_{\text{W2V}}\left(j \vert i \right) = \pi_j P_{\text{W2V}}\left(i \vert j \right).
\end{align}
In other words, the flow is symmetric (i.e., $T_{ij} = T_{ji}$) and $\pi_i$ is the stationary distribution for the Markov process.
Equation \eqref{eq:detailed_balanced} yields 
\begin{align}
    &\frac{\pi_i}{Z_i}\cdot \frac{s_i s_j}{\exp\left[d\left(\vec{u}_i, \vec{u}_j\right)^2 / 2\right]} = \frac{\pi_j}{Z_j}\cdot \frac{s_i s_j}{\exp\left[d\left(\vec{u}_i, \vec{u}_j\right)^2 / 2 \right]} \\
    &\iff \frac{\pi_i}{Z_i} = \frac{\pi_j}{Z_j} =: G', \label{eq:const}
\end{align}
where $G'$ is a constant. 
Substituting Eq.~\eqref{eq:model_node2vec_distance} into Eq.~\eqref{eq:flow} and exploiting Eq.~\eqref{eq:const} yield 
\begin{align}
    \label{eq:flow_gravity_model}
    T_{ij} &= \pi_i P_{\text{W2V}}\left(j \vert i \right)\nonumber \\
           &= \frac{\pi_i}{Z_i} \cdot \frac{s_i s_j}{\exp\left[d\left(\vec{u}_i, \vec{u}_j\right)^2 / 2 \right]}\nonumber \\ 
           &= G' \frac{s_i s_j}{\exp\left[d\left(\vec{u}_i, \vec{u}_j\right)^2 / 2 \right]}. 
\end{align}
Remind that $T_{ij}:= G M_i M_j f\left(d\left(\vec{u}_i,\vec{u}_j\right)\right)$ in the gravity model.
Thus, the word2vec is a gravity model with $M_i=s_i$ and $f\left(d\left(\vec{u}_i,\vec{u}_j\right)\right) = 1/\exp\left[d\left(\vec{u}_i,\vec{u}_j\right) ^ 2 /2\right]$.
Pulling force $s_i$ corresponds to the mass in the gravity model but does not have a solid physical meaning.

\section{Population, pulling force and gravitational potential}

What is pulling force $s_i$? Is it related to the mobile population $\pi_i$ in each point?
To answer the questions, we rewrite $\pi_i$ as a function of $s_i$ as follows.
With Eq.~\eqref{eq:const}, we rewrite $\pi_i$ as 
\begin{align}
    \pi_i &= G' Z_i \nonumber \\
        &= G' s_i \sum_{\ell=1}^N s_\ell f\left(d\left(\vec{u}_i,\vec{u}_\ell\right)\right)\\
        &= -G' s_i \Phi(\vec{u}_i), \label{eq:flow_gravity_model_2} \\ %\nonumber \\ 
\end{align}
where we have defined $\Phi(\vec{u})$ by 
\begin{align}
    \Phi(\vec{u}):=-G' \sum_{\ell=1}^N s_\ell f\left(d\left(\vec{u},\vec{u}_\ell\right)\right).
\end{align}
In other words, $\pi_i$ is proportional to the product of $s_i$ and $\Phi(\vec{u})$.
Therefore, points with a large $\pi_i$ tend to have a large $s_i$ value.
What does $\Phi(\vec{u})$ represent? 
Variable $\Phi(\vec{u})$ is analogous to the gravitational potential in mechanical physics, which takes form $-GM/d$, where $M$ is the mass and $d$ is the distance from the mass.
%In other words,  $\Phi(\vec{u})$ can be seen as the gravitational potential for the field, where 
%each point $\ell$ has a mass $s_{\ell}$ and is away from $\vec{u}$ with distance $d(\vec{u},\vec{u}_\ell)$, and the effect of the mass decays by the factor of $\exp\left[d(\vec{u},\vec{u}_\ell) ^2 /2\right]$.
Point $\vec{u}$ has a large $\Phi(\vec{u})$ if there are many points $\ell$ around $\vec{u}$ that have a large $\vec{s}_\ell$.
Potential $\Phi(\vec{u})$ is similar to the eigen centrality for networks that assigns a large centrality score for the node with many neighbors with a large centrality scores.


Population $\pi_i$ is given by $-s_i\Phi(\vec{u}_i)$. What does it mean for the mobility for scientists?
Suppose two institutions A and B. They have many influx from other institutions for different reasons:
(i) A is indeed an attractive institution which everyone on the planet wants to go but isolated from other institutions;
(ii) B is not attractive but close to many attractive institutions (e.g., small universities in a big city) and thus gain many influx from surrounding institutions.
Pulling force $s_i$ indicates the first factor, i.e., how attractive the individual institution is. 
Potential $\Phi(\vec{u})$ indicates how attractive the field is, accounting for the second factor.
The population is large when pulling force $s_i$ is large or the potential $\Phi(\vec{u}_i)$ is small. 

\section*{Random idea}
\begin{itemize}
    \item Compute the potential in the embedding and visualize it
    \item Does the potential predict the birth of new journals?
    \item What does $s_i$ indicate in the mobility data? Correlation to prestige?
    \item Use different kernel e.g., $f(d) = 1/exp(d)$, $f(d)=1/exp(\text{L1 norm})$ in the word2vec.
\end{itemize}
%\section*{Note}
%\begin{itemize}
%    \item The mass $m_i$ in this draft is different from $M_i$ in the manuscript. Variable $m_i$ is a parameter for the gravity model. 
%\end{itemize}

\printbibliography{}
   
\end{document} %}}}
